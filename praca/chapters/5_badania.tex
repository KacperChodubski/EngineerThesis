\chapter{BADANIA I ANALIZA WYNIKÓW}

tekst


\section{Ocena świadomości agentów}

jak dobrze agenci generują realistyczne odpowiedzi:\\
- czy agenci potrafią mówić o sobie\\
- czy pamiętają informacje o sobie, imię czy styl życia\\
- czy ich wypowiedzi odwołują się do przeszłych wydarzeń\\


\section{Porównanie wydajności modeli językowych}

- benchmark, ogólna ocena modeli i porównanie\\
- porównanie różnych modeli LLM używanych w aplikacji (np llama2, llama3)\\
- testy: czas odpowiedzi, czas generowania, zgodnosc z kontekstem, naturalnosc dialogow, zroznicowanie odpowiedzi


\section{Ocena jakości konwersacji przez użytkowników}

- subiektywna ocena modeli\\
- ankieta przeprowadzona wśród studentów, którzy ocenią jakość konwersacji generowanych przez różne modele\\
- która konwersacja była najbardziej naturalna i spójna, + uwagi i sugestie będzie można dodać

\section{Rozprzestrzenianie się informacji wśród agentów}

- jak w papierze\\
- eksperymenty związane z rozprzestrzenianiem się informacji między agentami\\
- do ilu agentów dociera informacja gdy na początku znał ją jeden agent (np. planowany ślub albo olimpiada)\\
- jak ważność informacji wpływa na jej propagację (np ślub i olimpiada ważne więc powinny się rozprzestrzenic, a to ze ktoś zjadł jajecznicę na śniadanie już nie)


\section{Wnioski i ograniczenia}

- wyzwania napotkane podczas pracy z aplikacją związane z modelami\\
- pewnie nie zawsze będzie się wyszukiwał ddobry kontekst, problemy z retrievalem\\
- halucynacje
