\chapter{TECHNOLOGIE, ALGORYTMY I NARZĘDZIA}

To jest rozdział.

\section{Podrozdziały}

W Latexu w klasie dokumentów \textbf{book} wyróżniamy rozdziały (\textbf{chapter}), podrozdziały \textbf{section}, podpodrozdziały \textbf{subsection}, podpodpodrozdziały \textbf{subsubsection} i paragrafy (\textbf{paragraph}). Podpodpodrozdziały i paragrafy domyślnie nie są numerowane ani nie występują w spisie treści. Zachowanie to można zmienić poprzez funkcję \textbf{setcounter} umieszczaną w preambule. Wykomentowany przykład można znaleźć w kodzie tego dokumentu.

Obecnie znajdujemy się na poziomie podrozdziału. Pozostałe przykłady poniżej.

\subsection{Podpodrozdział}

To jest podpodrozdział.

\subsubsection{Podpodpodrozdział}

To jest podpodpodrozdział. On nie jest domyślnie numerowany i nie występuje w spisie treści.

\paragraph{Paragraf}

A to jest paragraf. On również nie jest domyślnie numerowany i nie występuje w spisie treści.

