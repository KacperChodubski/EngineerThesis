\chapter{PROJEKT SYSTEMU}

To jest rozdział. Tu należy umieścić kilkuzdaniowe wprowadzenie do rozdziału.




\section{Interfejs gry}

To jest podrozdział.

\subsection{Podpodrozdział}

To jest podpodrozdział.

\subsubsection{Podpodpodrozdział}

To jest podpodpodrozdział. On nie jest domyślnie numerowany i nie występuje w spisie treści.

\paragraph{Paragraf}

A to jest paragraf. On również nie jest domyślnie numerowany i nie występuje w spisie treści.




\section{Architektura serwera i warstwy logicznej}

\section{Agenci}

\subsection{Pamięć agentów}
Pamięć agentów jest kluczowym elementem ich funkcjonowania.
To dzięki niej agenci reagują w sposób bardziej ludzki.

\subsection*{Architektura pamięci agentów}
Pamięć opiera się na modelu języka naturalnego typu encoder, który przekształca
nowe wspomnienia agenta na wektor. Później, w trakcie interakcji czy budowania nowych
wspomnień przez agenta jest wybierane k wspomnień, które uzyskały największą średnią
ważoną z trzech metryk: podobieństwa (relevance), ważności (importance) i niedawności (recency).
Wybrane wspomnienia są później przekazywane do modelu językowego typu decoder, który na
podstawie ich i danego wydarzenia decyduje o zachowaniu agenta.

\subsection*{Metryka podobieństwa}
Metryka podobieństwa odpowiada podobieństwu tematycznemu nowego wydarzenia do
wydarzenia zapisanego w pamięci agenta. Jest to uzyskiwane za pomocą obliczonej
odległości cosinusowej wektorów stworzonych przez model typu encoder.

\subsection*{Metryka ważności}
Metryka istoty jest uzyskiwana
poprzez ocenę ważności danego wydarzenia w kontekście życia agenta przez LLM.
Metryka odpowiada na pytanie jak ważne dla danego agenta w skali od 1 do 10 jest
dane wydarzenie np. pościelenie łóżka będzie miało skalę bliską 1, gdzie wzięcie
ślubu skalę bliską 10.

\paragraph{Metryka niedawności}
Metryka niedawności odpowiada mechanizmowi "zapominania" przez agenta.
Wspomnienia uzyskują wynik zgodnie ze wzorem:

\begin{equation}
	score = max\_score * decay\_factor ^ {time\_diff}
\end{equation}




\section{Komendy dla modeli językowych}

Jakość odpowiedzi generowanych przez duże modele językowe w znacznym stopniu zależy od zapytań, które są do nich wysyłane. Tekst przekazywany do modelu nazywany jest komendą (ang. \textit{prompt}), której główną częścią jest pytanie lub polecenie opisujące, co model powinien wykonać. W przypadku techniki RAG w skład komendy wchodzi także kontekst, czyli tekst ukazujący szerszą perspektywę danego tematu. W samym poleceniu zawarta jest także informacja dla modelu, aby ten, podczas odpowiadania na pytanie lub wykonywania postawionego mu zadania, korzystał bezpośrednio z dostarczonego mu kontekstu - jeżeli jakaś informacja w nim nie występuje, model powinien jasno zakomunikować, że nie jest w stanie odpowiedzieć na pytanie. Nie powinien próbować wymyślać odpowiedzi jedynie na podstawie własnej wiedzy.




\section{Komunikacja między serwisami}

To jest podrozdział.

\subsection{Podpodrozdział}

To jest podpodrozdział.

\subsubsection{Podpodpodrozdział}

To jest podpodpodrozdział. On nie jest domyślnie numerowany i nie występuje w spisie treści.

\paragraph{Paragraf}

A to jest paragraf. On również nie jest domyślnie numerowany i nie występuje w spisie treści.
