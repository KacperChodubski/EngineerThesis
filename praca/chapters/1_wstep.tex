\chapter{WSTĘP I CEL PRACY (\textit{Alicja Wagner})}
\label{chap:introduction}
% Wstępu wstępu - tutaj należy pokrótce opisać o co chodzi w pracy i wyraźnie wskazać cel pracy!

% Wstęp i cel pracy nakreśla problematykę opisaną lub rozwiązywaną w pracy dyplomowej wraz z uzasadnieniem celowości jej realizacji. Podaje cel i ewentualnie tezę (hipotezę). Syntetycznie opisuje dotychczasowe dokonania w danej tematyce, założenia techniczne oraz może zwięźle przedstawić zawartość poszczególnych rozdziałów. W przypadku pracy realizowanej przez kilku studentów, przy omawianiu zawartości rozdziałów należy podać ich autorów. Punkty stanowiące element składowy podrozdziału powinny być opracowane przez jednego autora

W ostatnich latach coraz częściej zaczęto wykorzystywać duże modele językowe (ang. \textit{LLM, Large Language Model}) w aplikacjach obejmujących różne sektory. Duże zainteresowanie sztuczną inteligencją zaowocowało jej szybkim rozwojem, co sprzyjało powstawaniu nowych pomysłów dla~zastosowań przetwarzania języka naturalnego.

Powstała także technika Retrieval-Augumented Generation (RAG), dzięki której możliwe jest podanie modelowi językowemu kontekstu, co zapobiega halucynacjom. Za jej pomocą można odwoływać się do poprzednich wiadomości lub zasilić model danymi, do których wcześniej nie miał dostępu. Otwiera to nowe możliwości tworzenia tekstu na podstawie konkretnych treści. Ponadto, jest to rozwiązanie o wiele tańsze i szybsze niż dotrenowywanie modeli, nie mówiąc już o trenowaniu ich od podstaw.

Symulowanie ludzkich zachowań staje się coraz bardziej realistyczne dzięki zastosowaniu innowacyjnych technologii. Można upodabniać zachowanie modeli do zachowania człowieka poprzez odwzorowywanie czynności takich jak percepcja, zpamiętywanie, planowanie i rozumowanie. Technika RAG ułatwia tworzenie angażujących i realistycznych scenariuszy.


\section{Motywacja}

% dlaczego akurat taki temat, czemu warto coś robić w tej dziedzinie

\subsection{Dynamiczny rozwój dziedziny}

W przeciągu ostatnich kilkudziesięciu lat dało się zaobserwować intensywny rozwój sztucznej inteligencji (ang. \textit{AI, Artificial Intelligence}). Na całym świecie zaczęto prowadzić coraz więcej badań dotyczących sztucznej inteligencji, co ma swoje odzwierciedlenie w statystykach cytowań prac związanych z tą tematyką. Jak pokazują przeglądy bibliometryczne, w latach 2012 - 2022 wzrost liczby publikacji w dziedzinie AI i Big Data był wykładniczy, z wyraźnym przyspieszeniem od 2019 roku\cite{p_v_thayyib_2023}. Sztuczna inteligencja zaczęła być wykorzystywana w wielu dyscyplinach, takich jak ochrona zdrowia, edukacja, biznes i zarządzanie, a także turystyka czy rozrywka. 

Jednym z podobszarów sztucznej inteligencji jest przetwarzanie języka naturalnego (ang. \textit{NLP, Natural Language Processing}). Dziedzina ta zajmuje się przekształcaniem języka zrozumiałego dla człowieka na taki, który jest zrozumiały dla komputera. Dzięki temu można w stosunkowo łatwy sposób analizować, generować i przetwarzać tekst. Znaczącą rolę odgrywają tutaj także duże modele językowe, które w ostatniej dekadzie dynamicznie się rozwijały. Największą popularność zyskały, gdy w listopadzie 2022 roku firma OpenAI uruchomiła ChatGPT-3.5. Czatbot ten od samego początku cieszył się ogromnym zainteresowaniem. W niespełna tydzień korzystało z niego ponad milion użytkowników\cite{ChatGPT2024}.

W dobrze przemyślanym i zbudownaym systemie możliwe jest nawet generowanie nowych poziomów gier wideo. Jednak jak w wielu aplikacjach wykorzystujących wytrenowane modele, efekty są mocno uzależnione od jakości danych\cite{Todd_2023}.


\section{Cel pracy}

co chcemy zrobić, implementacja papieru
