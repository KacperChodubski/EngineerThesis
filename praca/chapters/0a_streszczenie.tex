\chapter*{Streszczenie}
% \addcontentsline{toc}{chapter}{STRESZCZENIE}  
Praca skupia się na zagadnieniu wykorzystania modeli języka naturalnego. Większość obecnego rynku skupia się na bezpośrednim wykorzystaniu modeli do miedzy innymi pozyskiwania informacji, redagowania tekstów i tym podobnym. Nasz projekt wykorzystuje model jako narzędzie do imitowania ludzkiego postrzegania świata. Model jest wykorzystywany jako "osoba" podejmująca decyzje na wzór tych człowieka. Żeby móc zaprezentować działanie tego mechanizmu potrzebowaliśmy zbudować wirtualne środowisko, gdzie agenci, czyli nasz odpowiednik wirtualnego człowieka, będą podejmowali działania zaplanowane i zarządzane przez model językowy. Pozwoli nam to zobaczyć jak dużo wiedzy na temat realnego świata jest zawarte w zbiorach danych używanych przez twórców tych modeli oraz jak dobrze odzwierciedlają one zachowanie ludzi.
\newline
\newline
\textbf{Słowa kluczowe:} 
Agent, LLM
\newline
\textbf{Dziedzina nauki i techniki, zgodnie z wymogami OECD: }
\newline
Nauki o komputerach i informatyka


\chapter*{Abstract}
% \addcontentsline{toc}{chapter}{ABSTRACT}  
Abstract in english.
\newline
\newline
\textbf{Keywords:}
Keyword 1, Keyword 2, Keyword 3
\newline
\textbf{Field of science and technology in accordance with OECD requirements: } \newline
Computer Science and Information technology
