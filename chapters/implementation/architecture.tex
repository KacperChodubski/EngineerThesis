\chapter{Agenci}

\section{Pamięć agentów}
Pamięć agentów jest kluczowym elementem ich funkcjonowania.
To dzięki niej agenci reagują w sposób bardziej ludzki.

\subsection{Architektura pamięci agentów}
Pamięć opiera się na modelu języka naturalnego typu encoder, który przekształca
nowe wspomnienia agenta na wektor. Później, w trakcie interakcji czy budowania nowych
wspomnień przez agenta jest wybierane k wspomnień, które uzyskały największą średnią
ważoną z trzech metryk: podobieństwa (relevance), ważności (importance) i niedawności (recency).
Wybrane wspomnienia są później przekazywane do modelu językowego typu decoder, który na
podstawie ich i danego wydarzenia decyduje o zachowaniu agenta.

\subsubsection{Metryka podobieństwa}
Metryka podobieństwa odpowiada podobieństwu tematycznemu nowego wydarzenia do
wydarzenia zapisanego w pamięci agenta. Jest to uzyskiwane za pomocą obliczonej
odległości cosinusowej wektorów stworzonych przez model typu encoder.

\subsubsection{Metryka ważności}
Metryka istoty jest uzyskiwana
poprzez ocenę ważności danego wydarzenia w kontekście życia agenta przez LLM.
Metryka odpowiada na pytanie jak ważne dla danego agenta w skali od 1 do 10 jest
dane wydarzenie np. pościelenie łóżka będzie miało skalę bliską 1, gdzie wzięcie
ślubu skalę bliską 10.

\subsubsection{Metryka niedawności}
Metryka niedawności odpowiada mechanizmowi "zapominania" przez agenta.
Wspomnienia uzyskują wynik zgodnie ze wzorem:

\begin{equation}
	score = max\_score * decay\_factor ^ {time\_diff}
\end{equation}

