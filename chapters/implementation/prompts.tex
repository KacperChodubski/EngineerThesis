\chapter{PROMPTS}

To jest rozdział.

\section{Podrozdziały}

W Latexu w klasie dokumentów \textbf{book} wyróżniamy rozdziały (\textbf{chapter}), podrozdziały \textbf{section}, podpodrozdziały \textbf{subsection}, podpodpodrozdziały \textbf{subsubsection} i paragrafy (\textbf{paragraph}). Podpodpodrozdziały i paragrafy domyślnie nie są numerowane ani nie występują w spisie treści. Zachowanie to można zmienić poprzez funkcję \textbf{setcounter} umieszczaną w preambule. Wykomentowany przykład można znaleźć w kodzie tego dokumentu.

Obecnie znajdujemy się na poziomie podrozdziału. Pozostałe przykłady poniżej.

\subsection{Podpodrozdział}

To jest podpodrozdział.

\subsubsection{Podpodpodrozdział}

To jest podpodpodrozdział. On nie jest domyślnie numerowany i nie występuje w spisie treści.

\paragraph{Paragraf}

A to jest paragraf. On również nie jest domyślnie numerowany i nie występuje w spisie treści.

\section{Podstawowe elementy typograficzne}

\subsection{Twarda spacja}

Twarda spacja jest bardzo istotnym elementem, gdyż zabrania Latex'owi łamanie linii w miejscu jej wystąpienia, a tym samym pozwoli na niejako ,,sklejenie'' wyrazów ze sobą. Dzięki temu możemy uniknąć tzw.\ sierot (pojedynczych znaków na końcu wiersza). W Latex twardą spację umieszcza się wstawiając znak tyldy~(\textasciitilde). Zapisujemy to więc np.\ tak: ,,dokument, w{\textasciitilde}którym''.

\subsection{Formatowanie tekstu}

Aby zapewnić poprawny wygląd tekstu należy pamiętać o kilku rzeczach:

\begin{itemize}
 \item Linia poprzedzona procentem to komentarz.
 \item Poprzedzaniu spacji występującej po kropce kończącej skrót znakiem ucieczki, odstęp będzie wtedy taki, jak odstęp między wyrazami a~nie między zdaniami. Przykładowo zapis ,,np. tekst'' vs. ,,np.\ tekst''. Ten drugi jest poprawny, a zapisany został tak: ,,np.$\backslash$~tekst''.
 \item Skróty pisane wielkimi literami kończące zdanie powinny posiadać {$\backslash$}@ przed kropką kończącą zdanie, np.\ OCS{$\backslash$}@\@. Spowoduje to potraktowanie spacji jako spacji międzyzdaniowej z nie międzywyrazowej.
 \item Cudzysłowie zawsze tworzymy używając podwójnego przecinka jako symbolu otwierającego cudzysłów, oraz podwójnego apostrofu zamykającego cudzysłów.
 \item Kursywę uzyskujemy za pomocą słowa kluczowego {$\backslash$}textit, co w efekcie daje \textit{tekst kursywą}. Pogrubiony \textbf{używamy słowa kluczowego textbf}. Każdorazowo tekst mający być napisany danym krojem otaczamy nawiasami klamrowymi.
 \item Myślnik (--) tworzymy poprzez umieszczenie bezpośrednio po sobie dwu kresek (minusy). Różnica między nimi jest zasadnicza. Pojedynczy myślnik generuje krótką kreskę (-), podwójny długą (--), potrójny najdłuższą (---).
 \item Odwołania do różnych elementów dokumentu robimy poprzez słowo kluczowe \textbf{ref()}. Jako jego parametr wstawiamy nazwę zdefiniowaną za pomocą słowa kluczowego \textbf{label()}. Należy pamiętać, że odwołanie zwraca jedynie numer elementu, słowo opisowe, jak np.\ rozdział czy rysunek należy dodać samodzielnie. Polecam tutaj przyjąć jakąś konwencję i się jej trzymać w całym dokumencie. Tak samo należy postępować w przypadku etykiet.
 \item Latex doskonale radzi sobie z dzieleniem wyrazów na końcach linii, jednak czasami zachodzi konieczność wymuszenia podziału w określonym miejscu. W tym celu należy zastosować konstrukcję $\backslash$-. Latex takiego ukośnika nie wydrukuje dopóty, dopóki rzeczywiście w tym miejscu nie zostanie wykonane przeniesienie części wyrazu. Możliwe jest dodanie wielu podziałów w jednym wyrazie. Użyte wtedy zostanie to, które spowoduje wygenerowania ,,najładniejszego'' tekstu.
\end{itemize}

\section{Podział linii i paragrafy}
\label{podzial}

Nowy paragraf rozpoczyna się poprzez wstawienie jednej wolnej linii. Latex automatycznie wygeneruje wcięcie. Należy pamiętać, że pierwszy paragraf, zgodnie ze standardami drukarskimi, nie ma wcięcia! Możemy tym sterować za pomocą poleceń \textbf{noindent} (brak wcięcia) oraz \textbf{indent} (dodatkowe wcięcie).

Jeżeli chcemy po prostu zrobić nową linię, bez tworzenia nowego paragrafu używamy konstrukcji $\backslash\backslash$. Efekt będzie taki, że paragraf\\
będzie kontynuowany w nowej linii. Nie spowoduje to jednak rozciągnięcia poprzedniej linii. Zostanie ona przerwana tam gdzie tego sobie zażyczymy i kontynuowana w nowej linijce.

A co w przypadku, gdy chcemy z jakiegoś powodu przerwać linię, ale wymusić justowanie tekstu? Weźmy dla przykładu fragment:

Trzecim istotnym aspektem jest stosowana w~trakcie wytwarzania ontologii metodologia pracy~\cite{boinski2012kaskbook,boinski2011security}. Zastosowanie jednej z~uznawanych metodologii, takich jak Methontology, NeOn czy metodologia opracowana przez Noy~i~McGuiness, znacząco wpływa na jakoś uzyskanego produktu. Wspomniane metodologie w~dużej mierze uwzględniają potrzebę przyszłej integracji wiedzy, a~w połączeniu z~narzędziami typu Protégé czy OCS~\cite{boinski2007kaskbook,boinski2009ocs,boinski2010zespolowa}, pozwalają na tworzenie spójnych i~formalnie oraz logicznie poprawnych ontologii.

Tekst zostaje bardzo brzydko złamany w środku odnośników do cytowań. Użycie podwójnego po słowie ,,metodologia'' w pierwszym zdaniu ukośnika da nam natomiast taki efekt:

Trzecim istotnym aspektem jest stosowana w~trakcie wytwarzania ontologii metodologia \\ pracy~\cite{boinski2012kaskbook,boinski2011security}. Zastosowanie jednej z~uznawanych metodologii, takich jak Methontology, NeOn czy metodologia opracowana przez Noy~i~McGuiness, znacząco wpływa na jakoś uzyskanego produktu. Wspomniane metodologie w~dużej mierze uwzględniają potrzebę przyszłej integracji wiedzy, a~w połączeniu z~narzędziami typu Protégé czy OCS~\cite{boinski2007kaskbook,boinski2009ocs,boinski2010zespolowa}, pozwalają na tworzenie spójnych i~formalnie oraz logicznie poprawnych ontologii.

Też nie ładnie, gdyż linijka jest niewyjustowana. Z pomocą przychodzi nam tutaj komenda \textbf{linebreak[]}, gdzie w nawiasie kwadratowym podajemy liczbę od 1 do 4 określająca jak bardzo zależy nam na tym, by linia została złamana w tym miejscu (4 to najwyższa wartość). Efekt jest następujący:

Trzecim istotnym aspektem jest stosowana w~trakcie wytwarzania ontologii metodologia \linebreak[4] pracy~\cite{boinski2012kaskbook,boinski2011security}. Zastosowanie jednej z~uznawanych metodologii, takich jak Methontology, NeOn czy metodologia opracowana przez Noy~i~McGuiness, znacząco wpływa na jakoś uzyskanego produktu. Wspomniane metodologie w~dużej mierze uwzględniają potrzebę przyszłej integracji wiedzy, a~w połączeniu z~narzędziami typu Protégé czy OCS~\cite{boinski2007kaskbook,boinski2009ocs,boinski2010zespolowa}, pozwalają na tworzenie spójnych i~formalnie oraz logicznie poprawnych ontologii.

Jeżeli z jakiegoś powodu potrzebujemy nową linię to używamy komendy \textbf{newpage}. \newpage Tekst występujący po niej znajdzie się na nowej stronie. Rozdziały itp.\ automatycznie generują nową stronę, przy czym w układzie dwustronnym nowy rozdział zawsze zacznie się od nieparzystej strony.

\section{Środowisko matematyczne}

Środowisko matematyczne otwieramy i zamykamy znakiem \$. Niektóre funkcje można używać tylko wewnątrz takiego środowiska. Przykładem niech będzie funkcja \textbf{mathcal} zamieniająca duże litery w symbole o charakterystycznym kroju, stosowanym do opisywania stałych, np.\ $\mathcal{O}$ czy $\mathcal{R(D,P,T,S,U,I)}$. Pamiętać należy, że zamienione zostaną wszystkie litery w wyrażeniu występującym wewnątrz nawiasów klamrowych.

Niektóre konstrukcje, np.\ równania, automatycznie włączają tryb matematyczny. Równania dobrze jest opisać, przykład przedstawia Równanie~\ref{eq:przyklad}.

\begin{equation}
  \mathcal{O(K,B,C,R)}
  \label{eq:przyklad}
\end{equation}

gdzie:\\
$\mathcal{K}$ - zbiór klas wchodzących w~skład ontologii,\\
$\mathcal{B}$ - zbiór bytów wchodzących w~skład ontologii,\\
$\mathcal{C}$ - zbiór komentarzy przypisanych do klas $\mathcal{K}$ i~bytów $\mathcal{B}$ wchodzących w~skład ontologii,\\
$\mathcal{R}$ - zbiór relacji wiążących elementy ontologii.


W równaniach możemy stosować różne dodatkowe symbole oraz np.\ wyrównywać je do określonego miejsca. Służy do tego blok typu \textbf{split}, a sam punkt wyrównania określony jest ampersandem (\&). Przykład zastosowania prezentuje równanie~\ref{eq:split_ex}.

\begin{equation}
  \label{eq:split_ex}
  \begin{split}
    \forall {x_1 \leq y_1, x_2 \leq y_2}&: f(x_1+x_2,y_1+y_2)\\ 
				  &= \frac{y_1}{y_1+y_2}f(x_1,y_1)+\frac{y_2}{y_1+y_2}f(x_2,y_2)
  \end{split}
\end{equation}

\subsection{Twierdzenia i dowody}

Linie 75 -- 93 nagłówka dokumentu definiują nowe nazwy sekcji twierdzeń i dowodów, oraz znacznik końca dowodu (taki czarny kwadracik). Dzięki nim można uzyskać ładnie wyglądające twierdzenia jak poniżej (Twierdzenie~\ref{eq:lin:theorem}). Zauważmy, że równanie dowodu nie jest równaniem numerowanym. Wszędzie tam, gdzie nie chcemy by rozdział czy dowolna inna sekcja była numerowana należy w jej nazwie użyć gwiazdki, np. \textbf{$\backslash$begin\{equation*\}}.

\begin{theorem}
 Podobieństwo pomiędzy pojęciami $A$ i~$B$ opisane jest stosunkiem ilości informacji niezbędnej do opisania ich wspólności znaczeniowej oraz ilością informacji niezbędnej do ich opisania (Równanie~\ref{eq:lin:theoremeq}).

 \begin{equation}
   sim_{lin}(A,B)=\frac{\log P(common(A,B))}{\log P(description(A,B))}
   \label{eq:lin:theoremeq}
 \end{equation}
 \label{eq:lin:theorem}
\end{theorem}

\begin{proof}
  \begin{equation*}
   \begin{split}
     f(x,y)&=f(x+0,y+(y-x))\\
	   &=\frac{x}{y}*f(x,x) +~\frac{y-z}{x}*f(0,y-z)\\
	   &=\frac{x}{y}*1 +~\frac{y-z}{x}*0\\
	   &=\frac{x}{y} \qed
   \end{split}
 \end{equation*}
\end{proof}

Inna ciekawa konstrukcja wykorzystująca tryb matematyczny do zapisania pewnego stwierdzenia:

Niech $A \subseteq T$, $C = N_y(A) \neq W$, a~$\alpha_y = \min_{a \in A, b\notin C} \{y(a) +~y(b) - q(a, b)\}$ oraz

\[ y'(v) = \left\{ \begin{array}{ll}
                y(v) - \alpha_y & \mbox{jeżeli $v \in A$} \\
                y(v) +~\alpha_y & \mbox{jeżeli $v \in C$} \\
                y(v)            & \mbox{w innych przypadkach}
               \end{array}
       \right. \]

Zapis ten, acz skomplikowany, pozwala na reprezentację złożonych reguł matematycznych w postaci ładnie ułożonych i wyrównanych wierszy. Reguły \textbf{left} oraz \textbf{right} pozwalają na utworzenie nawiasów klamrowych, których rozmiar będzie automatycznie dostosowywany do rozmiaru elementu, jakie mają zawierać.
